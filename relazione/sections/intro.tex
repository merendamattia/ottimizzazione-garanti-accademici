\section{Introduzione}
\label{sec:introduction}

L'assegnazione dei garanti nei corsi universitari costituisce una 
questione fondamentale per la gestione ottimale delle risorse accademiche. 
Nell'ambito universitario, il \textit{garante} è un docente responsabile 
di rappresentare e tutelare la qualità didattica di un corso, garantendo 
il rispetto dei requisiti disciplinari e istituzionali. I garanti possono 
appartenere a diverse categorie contrattuali: docenti a tempo indeterminato, 
docenti a tempo determinato e, in casi eccezionali, docenti a contratto.

La sfida principale consiste nel soddisfare i vincoli ministeriali 
relativi ai garanti, garantendo al contempo un'allocazione equilibrata 
e sostenibile delle risorse. Ogni corso deve essere supportato da un 
numero minimo di garanti, suddivisi tra le diverse fasce contrattuali, 
per assicurare un livello adeguato di competenza e rappresentatività. 
Inoltre, è indispensabile che almeno il 50\% dei garanti afferisca al 
Settore Scientifico Disciplinare (SSD) caratterizzante del corso, al 
fine di garantire la coerenza tra l'offerta formativa e le competenze 
disciplinari.

Un'ulteriore complessità è rappresentata dall'impiego di docenti a 
contratto, il cui utilizzo deve essere limitato e subordinato alle 
sole situazioni in cui non sia possibile soddisfare i requisiti 
attraverso i docenti strutturati. La necessità di rispettare questi 
vincoli, combinata con la disponibilità limitata di personale e la 
necessità di bilanciare il carico di lavoro, rende questo problema 
una sfida organizzativa e computazionale significativa.

Un primo ostacolo affrontato in questo progetto riguarda la fase di 
pre-elaborazione dei dati forniti dall'università, descritta nella 
Sezione~\ref{sec:pre-proc} di questo elaborato. I dati risultano 
disomogenei e incompleti, richiedendo una significativa pulizia e 
riorganizzazione prima di poter essere utilizzati efficacemente nel 
modello ASP. Questa fase ha richiesto lo sviluppo di strumenti 
dedicati per uniformare e validare i dati.

La Sezione~\ref{sec:asp} esplora il cuore del nostro approccio, 
ovvero la costruzione del modello ASP. Qui abbiamo implementato 
regole logiche per soddisfare i vincoli ministeriali e massimizzare 
l'efficienza dell'assegnazione dei garanti, tenendo conto delle 
limitazioni sulle fasce contrattuali e sull'impiego dei docenti a contratto.

Successivamente, nella Sezione~\ref{sec:esempio-giocattolo}, presentiamo 
i risultati ottenuti su un esempio ridotto, un test che ha permesso 
di validare il modello in un ambiente controllato e di analizzare la qualità 
delle soluzioni generate. Infine, la Sezione~\ref{sec:expval} descrive 
l'applicazione del modello su un dataset completo contenente tutti i corsi 
universitari, eseguendo un benchmark su larga scala per valutare la 
capacità del nostro approccio di risolvere problemi reali e complessi.