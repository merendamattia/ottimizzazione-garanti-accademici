\section{Conclusioni}
\label{sec:conclusion}

Il lavoro presentato ha dimostrato l'efficacia di un sistema automatizzato per 
l'assegnazione dei garanti accademici, basato su Answer Set Programming. 
Attraverso l'integrazione di tecniche di preprocessing dei dati, la definizione 
accurata di vincoli ministeriali e la gestione delle priorità, è stato possibile 
sviluppare un modello flessibile e scalabile, in grado di affrontare le sfide 
legate alla gestione delle risorse accademiche.

I risultati ottenuti hanno evidenziato la capacità del modello di soddisfare i 
requisiti istituzionali, rispettando vincoli stringenti e massimizzando l'efficienza 
nell'uso delle risorse. L'analisi su diversi dataset, dai casi studio ridotti ai 
benchmark su larga scala, ha permesso di validare il sistema sia in ambienti 
controllati che su scenari più complessi e realistici. In particolare, i test 
sul dataset completo hanno evidenziato come il modello possa adattarsi a situazioni 
critiche, sebbene l'aggiunta di pesi flessibili abbia comportato un aumento 
esponenziale della complessità computazionale.