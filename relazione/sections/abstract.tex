\begin{abstract}
    Questo lavoro presenta l'analisi e l'implementazione di un sistema automatizzato per 
    l’assegnazione dei garanti ai corsi universitari, in conformità ai requisiti ministeriali. 
    L’obiettivo principale è garantire che ogni corso soddisfi i vincoli minimi di docenza, 
    rispettando le regole di distribuzione tra diverse categorie di docenti e ottimizzando 
    l’uso delle risorse disponibili.

    Utilizzando la programmazione logica con Answer Set Programming (ASP) \cite{ASP}, 
    abbiamo modellato il problema attraverso fatti, regole e vincoli derivati dai dati 
    ministeriali e universitari. Abbiamo implementato una serie di vincoli per rispettare 
    i minimi richiesti di docenti per corso, evitando sovrapposizioni improprie tra gli 
    incarichi dei docenti e considerando scenari realistici in cui un docente può assumere 
    più ruoli parziali.
    
    L’approccio è stato testato su un dataset reale contenente informazioni su corsi, SSD 
    (Settori Scientifico Disciplinari) e docenti dell'Università degli Studi di Parma. 
    I risultati dimostrano come il sistema possa trovare configurazioni ottimali che 
    soddisfano i requisiti, massimizzando l’efficienza e mantenendo flessibilità 
    nell’assegnazione dei docenti.

\end{abstract}