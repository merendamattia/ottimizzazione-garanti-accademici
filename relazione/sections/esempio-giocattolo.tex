\section{Esempio giocattolo}
\label{sec:esempio-giocattolo}


In the \textcolor{teal}{\textit{Introduction}}, you should start by providing background and context for your study, highlighting the importance and relevance of the topic. 
Then, clearly identify the research problem or gap in your study's existing literature, explaining why it is significant. 
State the research questions or objectives your study aims to answer, ensuring they are directly linked to the identified problem. 
Justify the need for your research by discussing its potential contributions or impact on the field.
You may also include a brief overview of the methodology, especially if it's novel or crucial to your study's contribution. 
Additionally, define the scope and limitations of your research, clarifying what the study will and will not cover.
If applicable, present your main thesis statement or hypothesis. 
Optionally, you can conclude the introduction with a brief outline of the paper's structure to guide the reader.