\section{Validazione della soluzione proposta}
\label{sec:expval}

Per testare l'efficacia e la correttezza del modello sviluppato, è stato creato un esempio 
giocattolo basato sui corsi di laurea in Informatica (LT) e Scienze Informatiche (LM). 
Questo dataset ridotto è stato progettato per verificare la capacità del modello ASP di 
generare soluzioni valide, rispettando i vincoli definiti e implementando correttamente 
le ottimizzazioni.

I dati di input per il modello sono stati generati attraverso un comando Python, che ha 
generato i file necessari contenenti i fatti relativi ai docenti, alle coperture, ai 
docenti a contratto e ai limiti ministeriali.

\begin{lstlisting}[language=bash]
 python3 main.py --3027 --5069
\end{lstlisting}

Una volta generati i file, l'analisi è stata eseguita utilizzando Clingo.

\begin{lstlisting}[language=bash]
 clingo -n 0 --parallel-mode 8 --time-limit=180 lp/* main.lp
\end{lstlisting}

Il modello ha prodotto una soluzione ottima in un \textit{centesimo di secondo}, 
dimostrando tempi di computazione estremamente brevi. Per il corso di Informatica 
(codice: 3027) sono stati assegnati 9 garanti in conformità ai requisiti, mentre per il corso di 
Scienze Informatiche (codice: 5069) sono stati assegnati 6 garanti, sempre in conformità ai 
requisiti.

È stata inoltre verificata la coerenza tra i garanti assegnati e gli SSD caratterizzanti, 
con 5 docenti afferenti per Informatica e 4 per Scienze Informatiche.

\subsection{Dataset ridotto: Dipartimento SMFI}
\label{sec:-dataset-dipartimento-smfi}

Per valutare la scalabilità del modello, è stato eseguito un test utilizzando un dataset 
ridotto relativo al Dipartimento di Scienze Matematiche, Fisiche e Informatiche (SMFI), 
comprendente sei corsi di studio. Anche in questo caso, i file di input sono stati generati 
tramite Python.

\begin{lstlisting}[language=bash]
 python3 main.py --3027 --5069 --3026 --5036 --3030 --5037
\end{lstlisting}

L'analisi è stata eseguita con Clingo utilizzando lo stesso comando precedentemente indicato, 
ma con un timer impostato a un'ora. Sebbene non sia stata trovata una soluzione ottimale 
entro il limite di tempo, il modello ha generato 23 soluzioni in pochissimi secondi, 
tra cui un modello ottimale rispetto ai vincoli soddisfacibili. Questo modello ottimale, 
tra quelli generati, ha assegnato zero docenti a contratto come garanti su un totale di 45 
garanti, dimostrando la capacità del sistema di rispettare i vincoli più stringenti.

Durante il test, è stata utilizzata la configurazione che prevede solo pesi pari a 10, al 
fine di ridurre la complessità computazionale. Tuttavia, il modello supporta anche l'opzione 
di pesi pari a 5, che, pur aumentando significativamente i tempi di computazione, offre 
maggiore flessibilità nella generazione delle soluzioni.


\subsection{Dataset grande: tutti i dipartimenti}
\label{sec:dataset-tutti-dipartimenti}

Come ultima valutazione del modello, è stato utilizzato un dataset comprendente tutti i corsi 
di studio dell'ateneo. Prima dell'elaborazione, sono stati esclusi i casi particolari relativi 
ai corsi inter-atenei, poiché non erano considerati rilevanti per questa analisi.
Anche in quest'ultimo caso, i file di input sono stati generati tramite Python.
\begin{lstlisting}[language=bash]
 python3 main.py --all
\end{lstlisting}

Durante questa valutazione, è stato impostato un limite di tempo di un'ora. Sebbene non sia stato 
trovato un modello ottimale entro il limite stabilito, il modello ha generato 283 soluzioni, 
tra cui un modello ottimale rispetto ai vincoli soddisfacibili. Questo modello finale includeva 
solo 11 docenti a contratto su un totale di 869 garanti assegnati, dimostrando una notevole 
efficienza nell'ottimizzazione delle risorse e una forte preferenza per l'utilizzo di docenti non a contratto.
