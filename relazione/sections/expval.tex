\section{Validazione della soluzione proposta}
\label{sec:expval}

Per testare l'efficacia e la correttezza del modello sviluppato, è stato creato un esempio 
giocattolo basato sui corsi di laurea in Informatica (LT) e Scienze Informatiche (LM). 
Questo dataset ridotto è stato progettato per verificare la capacità del modello ASP di 
generare soluzioni valide, rispettando i vincoli definiti e implementando correttamente 
le ottimizzazioni.

I dati di input per il modello sono stati generati attraverso un comando Python, che ha 
prodotto i file necessari contenenti i fatti relativi ai docenti, alle coperture, ai 
docenti a contratto e ai limiti ministeriali.

\begin{lstlisting}[language=bash]
 python3 main.py --3027 --5069
\end{lstlisting}

Una volta generati i file, l'analisi è stata eseguita utilizzando Clingo.

\begin{lstlisting}[language=bash]
 clingo -n 15 --parallel-mode 8 --time-limit=180 lp/* main.lp
\end{lstlisting}

Il modello ha prodotto una soluzione ottima in un \textit{centesimo di secondo}, 
dimostrando tempi di computazione estremamente brevi. Per il corso di Informatica 
(codice: 3027) sono stati assegnati 9 garanti in conformità ai requisiti, mentre per il corso di 
Scienze Informatiche (codice: 5069) sono stati assegnati 6 garanti, sempre in conformità ai 
requisiti.

È stata inoltre verificata la coerenza tra i garanti assegnati e gli SSD caratterizzanti, 
con 5 docenti afferenti per Informatica e 4 per Scienze Informatiche.

\subsection{Dataset ridotto: Dipartimento SMFI}
\label{sec:-dataset-dipartimento-smfi}

Per valutare la scalabilità del modello, è stato eseguito un test utilizzando un dataset 
ridotto relativo al Dipartimento di Scienze Matematiche, Fisiche e Informatiche (SMFI), 
comprendente sei corsi di studio. Anche in questo caso, i file di input sono stati generati 
tramite Python.

\begin{lstlisting}[language=bash]
 python3 main.py --3027 --5069 --3026 --5036 --3030 --5037
\end{lstlisting}

L'analisi è stata eseguita con Clingo utilizzando lo stesso comando precedentemente 
indicato. Il modello ha dimostrato di poter gestire un numero maggiore di corsi, producendo 
soluzioni valide in tempi brevi. In questo caso, il tempo di computazione è stato di circa 
\textit{due secondi} per ottenere la soluzione ottimale.

Durante il test, è stata utilizzata la configurazione che prevede solo pesi pari a 10, al 
fine di ridurre la complessità computazionale. Tuttavia, il modello supporta anche l'opzione 
di pesi pari a 5, che, pur aumentando significativamente i tempi di computazione, offre 
maggiore flessibilità nella generazione delle soluzioni.


\subsection{Dataset grande: tutti i dipartimenti}
\label{sec:dataset-tutti-dipartimenti}

Come ultima valutazione del modello, è stato utilizzato un dataset 
comprendente tutti i corsi di studio del dipartimento. Prima dell'elaborazione, sono stati 
esclusi i casi particolari relativi ai corsi inter-atenei, poiché non erano considerati 
rilevanti per questa analisi. Inoltre, è stato rimosso il vincolo che richiedeva che almeno 
il 50\% dei garanti afferissero al macrosettore caratterizzante del corso. Questa scelta è 
stata necessaria perché in molti corsi la composizione dei docenti non permetteva di 
soddisfare tale requisito: in alcuni casi, solo due docenti su venti erano afferenti al 
macrosettore caratterizzante, rendendo impossibile il rispetto di questa condizione.

La prima analisi è stata condotta utilizzando la configurazione con solo pesi pari a 10, 
ma questa si è rivelata insoddisfacibile a causa di due corsi specifici, rispettivamente 
il corso a ciclo unico di Farmacia (codice: 5079) e il corso a ciclo unico di Chimica e 
Tecnologia Farmaceutiche (codice: 5080). Rimossi questi corsi problematici, il modello 
è stato rilanciato con la configurazione basata esclusivamente su pesi pari a 10, 
risultando soddisfacibile in pochi secondi. Questa prima fase ha dimostrato che, 
eliminando i corsi che introducono situazioni particolarmente complesse, il modello 
è in grado di produrre soluzioni valide e coerenti in modo estremamente rapido.

Successivamente, per includere anche i corsi precedentemente rimossi, è stata 
effettuata una nuova run abilitando l'opzione dei pesi pari a 5. 
Entrambe le configurazioni sono state lanciate con il seguente comando:

\begin{lstlisting}[language=bash]
 clingo -n 15 --parallel-mode 10 --time-limit=3000 lp/* main.lp
\end{lstlisting}

Nonostante non sia stato raggiunto un modello ottimale entro il limite di tempo di un'ora, 
sono stati generati 10 modelli, tra cui uno ottimale rispetto ai vincoli soddisfacibili. 
L'aggiunta del peso 5 ha aumentato significativamente la flessibilità del modello, ma ha 
anche comportato un'esplosione esponenziale del costo computazionale. Questa crescita è 
stata causata dal maggiore numero di combinazioni generate, che hanno richiesto un tempo di 
elaborazione considerevolmente superiore.
