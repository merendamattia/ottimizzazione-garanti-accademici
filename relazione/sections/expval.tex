\section{Validazione della soluzione proposta}
\label{sec:expval}

Per testare l'efficacia e la correttezza del modello sviluppato, è stato creato un esempio 
giocattolo basato sui corsi di laurea in Informatica (LT) e Scienze Informatiche (LM). 
Questo dataset ridotto è stato progettato per verificare la capacità del modello ASP di 
generare soluzioni valide, rispettando i vincoli definiti e implementando correttamente 
le ottimizzazioni.

I dati di input per il modello sono stati generati attraverso un comando Python, che ha 
prodotto i file necessari contenenti i fatti relativi ai docenti, alle coperture, ai 
docenti a contratto e ai limiti ministeriali.

\begin{lstlisting}[language=bash]
 python3 main.py --3027 --5069
\end{lstlisting}

Una volta generati i file, l'analisi è stata eseguita utilizzando Clingo.

\begin{lstlisting}[language=bash]
 clingo -n 15 --parallel-mode 8 --time-limit=180 lp/* main.lp
\end{lstlisting}

Il modello ha prodotto una soluzione ottima in un \textit{centesimo di un secondo}, 
dimostrando tempi di computazione estremamente brevi. Per il corso di Informatica 
(codice: 3027) sono stati assegnati 9 garanti in conformità ai requisiti. Per il corso di 
Scienze Informatiche (codice: 5069) sono stati assegnati 6 garanti, sempre in conformità ai 
requisiti

È stata inoltre verificata la coerenza tra i garanti assegnati e gli SSD caratterizzanti, 
con 5 docenti afferenti per Informatica e 4 per Scienze Informatiche.

\subsection{Dataset ridotto: Dipartimento SMFI}
\label{sec:-dataset-dipartimento-smfi}

Per valutare la scalabilità del modello, è stato eseguito un test utilizzando un dataset 
ridotto relativo al Dipartimento di Scienze Matematiche, Fisiche e Informatiche (SMFI), 
comprendente sei corsi di studio. Anche in questo caso, i file di input sono stati generati 
tramite Python.

\begin{lstlisting}[language=bash]
 python3 main.py --3027 --5069 --3026 --5036 --3030 --5037
\end{lstlisting}

L'analisi è stata eseguita con Clingo utilizzando lo stesso comando precedentemente 
indicato. Il modello ha dimostrato di poter gestire un numero maggiore di corsi, producendo 
soluzioni valide in tempi brevi. In questo caso, il tempo di computazione è stato di circa 
\textit{due secondi} per ottenere la soluzione ottimale.

Durante il test, è stata utilizzata la configurazione che prevede solo pesi pari a 10, al 
fine di ridurre la complessità computazionale. Tuttavia, il modello supporta anche l'opzione 
di pesi pari a 5, che, pur aumentando significativamente i tempi di computazione, offre 
maggiore flessibilità nella generazione delle soluzioni.




\subsection{Dataset grande (tutti i dipartimenti)}
\label{sec:dataset-tutti-dipartimenti}