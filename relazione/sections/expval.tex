\section{Valutazione della soluzione proposta}
\label{sec:expval}

\subsection{Esempio giocattolo}
\label{sec:-dataset-giocattolo}

\mmerenda{magari cambiare nome della sezione in "Validazione della soluzione proposta", lo rende più professionale e accattivante}

Per validare il modello sviluppato e testare l’efficacia dell’approccio proposto,
abbiamo costruito un esempio giocattolo sfruttando il corso di laurea triennale di informatica
ed il corso di laurea magistrale in Scienze Informatiche.

Il dataset generato ci ha accompagnato nel processo di testing del codice ASP proposto;
permettendoci di verificare se fosse in grado di gestire correttamente la generazione dei termini,
i vincoli e le preferenze specificate.

I risultati ottenuti sono stati utilizzati per ottimizzare ulteriormente il codice,
permettendoci di identificare eventuali aree di miglioramento relative alla gestione
dei vincoli e all'efficienza della generazione dei termini.

\mmerenda{spiegare i comandi utilizzati per generare il dataset (il comando python)} \\
\mmerenda{specificare il comando utilizzato per lanciare l'analisi con clingo} \\
\mmerenda{stampare l'output andando a commentare il risultato}

\subsection{Dataset piccolo (dipartimento SMFI)}
\label{sec:-dataset-dipartimento-smfi}

\subsection{Dataset grande (tutti i dipartimenti)}
\label{sec:dataset-tutti-dipartimenti}