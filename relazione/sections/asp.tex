\section{ASP}\label{sec:asp}

In questa sezione approfondiamo l'organizzazione del progetto, evidenziando 
come sia stato strutturato per garantire chiarezza e modularità. 
Per facilitare l'ordine visivo e semplificare eventuali ispezioni manuali, 
il progetto è stato suddiviso in più file sorgenti, ciascuno dedicato a un 
aspetto specifico del dataset e delle regole.

Abbiamo separato i fatti e le regole in due file principali: uno dedicato 
ai docenti (Sezione~\ref*{sec:rules-docenti}) e un altro focalizzato sulle 
coperture (Sezione~\ref*{sec:rules-coperture}). Per maggiore praticità, è 
stato creato un file specifico per i docenti a contratto 
(Sezione~\ref*{sec:rules-docenti-contratto}) e un altro per i limiti 
ministeriali (Sezione~\ref*{sec:rules-ministeriale}).

Il file principale, il main, integra tutte le regole relative ai garanti, 
inclusi i vincoli e le priorità definite per l'ottimizzazione del sistema 
(Sezione~\ref*{sec:constraints}).

È importante sottolineare che tutti questi file vengono generati dinamicamente 
dal programma di preprocessing scritto in Python. Questa fase, gestita dal 
file main del preprocessing, automatizza la creazione delle regole e dei 
fatti ASP in base ai dati forniti, garantendo così una configurazione 
personalizzata e facilmente aggiornabile. 


\subsection{Docenti}\label{sec:rules-docenti}



\begin{lstlisting}[language=Python, caption={Esempio di codice Python}]
def saluta(nome):
    print(f"Ciao, {nome}!")

saluta("Mondo")
\end{lstlisting}


\subsection{Coperture}\label{sec:rules-coperture}
\subsection{Docenti a contratto}\label{sec:rules-docenti-contratto}
\mmerenda{spiegare qui che li usiamo come jolly}
\subsection{Ministeriale}\label{sec:rules-ministeriale}


\subsection{Vincoli}\label{sec:constraints}
\subsection{Gestione delle priorità}\label{sec:priorita}
\subsection{Ottimizzzazioni}\label{sec:optimizations}