\section{Answer Set Programming}\label{sec:asp}

In questa sezione approfondiamo l'organizzazione del progetto, evidenziando 
come sia stato strutturato per garantire chiarezza e modularità. 
Per facilitare l'ordine visivo e semplificare eventuali ispezioni manuali, 
il progetto è stato suddiviso in più file sorgenti, ciascuno dedicato a un 
aspetto specifico del dataset e delle regole.

Abbiamo separato i fatti e le regole in due file principali: uno dedicato 
ai docenti (Sezione~\ref*{sec:rules-docenti}) e un altro focalizzato sulle 
coperture (Sezione~\ref*{sec:rules-coperture}). Per maggiore praticità, è 
stato creato un file specifico per i docenti a contratto 
(Sezione~\ref*{sec:rules-docenti-contratto}) e un altro per i limiti 
ministeriali (Sezione~\ref*{sec:rules-ministeriale}).

Il file principale, il main, integra tutte le regole relative ai garanti, 
inclusi i vincoli e le priorità definite per l'ottimizzazione del sistema 
(Sezione~\ref*{sec:garanti}).

È importante sottolineare che tutti questi file vengono generati dinamicamente 
dal programma di preprocessing scritto in Python. Questa fase, gestita dal 
file \texttt{main.py} del preprocessing, automatizza la creazione delle regole 
e dei fatti ASP in base ai dati forniti, garantendo così una configurazione 
personalizzata e facilmente aggiornabile. 


\subsection{Docenti}\label{sec:rules-docenti}
Il file \texttt{docenti.lp} contiene una rappresentazione strutturata delle 
informazioni relative ai docenti, organizzate in diverse sezioni per garantire 
chiarezza e modularità. Questo file serve come base per definire le caratteristiche 
dei docenti e il loro SSD.

La prima parte del file definisce gli SSD disponibili, ognuno rappresentato da una 
coppia di valori: la disciplina e il livello associato (e.g., INF/01, MAT/03). Gli SSD 
sono utilizzati per verificare la compatibilità tra i docenti e i corsi di laurea.

Successivamente, vengono dichiarate le fasce contrattuali dei docenti (\texttt{td} 
per tempo determinato, \texttt{ti} per tempo indeterminato). Queste fasce sono 
utilizzate per distinguere i docenti in base alla loro tipologia di contratto,
aspetto rilevante per rispettare i requisiti ministeriali.

La sezione relativa ai docenti include l'elenco completo dei docenti, ciascuno 
identificato da una matricola unica. Per ogni docente, vengono indicate la fascia 
contrattuale e il settore scientifico disciplinare caratterizzante, che ne 
definisce il dominio di competenza. Queste informazioni sono formalizzate tramite 
il predicato \texttt{docente/3}, che associa la matricola del docente, la fascia, 
il settore disciplinare e il livello SSD.

\begin{lstlisting}[language=prolog, caption={Esempio struttura dati di \texttt{docenti.lp}.}]    
 %  SEZIONE: SSD
 ssd(inf).

 %  SEZIONE: FASCIE
 fascia(td).
 fascia(ti).

 %  SEZIONE: Docenti
 %  Rossi Mario (1234), SSD caratterizzante: inf
 matricola_docente(1234).

 %  SEZIONE: SSD caratterizzante dei docenti
 %  Rossi Mario (1234), SSD caratterizzante: inf
 docente(1234, td, inf) :- matricola_docente(1234), fascia(td), ssd(inf).
\end{lstlisting}


\subsection{Coperture}\label{sec:rules-coperture}
Il file \texttt{coperture.lp} è fondamentale per rappresentare la struttura dei 
corsi universitari, le informazioni sui docenti assegnati e le caratteristiche 
associate a ciascun corso. Questo file contiene diverse sezioni organizzate per 
garantire chiarezza e modularità nella definizione delle informazioni.

La prima parte del file introduce i tipi di corso (\texttt{laurea/1}) e i relativi 
SSD utilizzati per caratterizzare i corsi e i docenti. Gli SSD caratterizzanti di 
un corso vengono estratti dinamicamente dagli SSD degli insegnamenti di quel corso 
che sono classificati come TAF A. Questa scelta consente di identificare in modo 
coerente i macrosettori scientifico-disciplinari fondamentali per ciascun corso, 
basandosi sui suoi insegnamenti di base. Successivamente, vengono definiti i TAF 
che classificano ulteriormente gli insegnamenti.

Le informazioni sui corsi sono strutturate in due livelli: i codici identificativi 
di ciascun corso (\texttt{codice\_corso/1}) e la loro descrizione dettagliata 
tramite il predicato \texttt{corso/3}. Quest'ultimo associa il codice del corso, 
il tipo di laurea, il SSD caratterizzante e il livello del settore disciplinare.

Un'altra parte importante del file riguarda la relazione tra corsi e docenti, 
formalizzata tramite il predicato \texttt{cattedra/4}. Questo predicato associa 
un docente (identificato tramite la matricola) a un corso specifico, includendo 
il tipo di laurea e il TAF relativo. 

Di seguito, un esempio rappresentativo del contenuto del file.

\begin{lstlisting}[language=prolog, caption={Esempio struttura dati di \texttt{coperture.lp}.}]    
 %  SEZIONE: Tipi di Corso
 laurea(lt).
 laurea(lm).

 %  SEZIONE: TAF
 taf(a).
 taf(b).
 taf(c).

 %  SEZIONE: Corsi
 %  INFORMATICA (3027)
 codice_corso(3027).

 %  SEZIONE: Informazioni Corsi
 corso(3027, lt, fis) :- codice_corso(3027), laurea(lt), ssd(fis).
 corso(3027, lt, mat) :- codice_corso(3027), laurea(lt), ssd(mat).
 corso(3027, lt, inf) :- codice_corso(3027), laurea(lt), ssd(inf).
 
 %  SEZIONE: Relazioni Corsi-Docenti
 %  Corso: 3027, Docente: Rossi Mario
 cattedra(3027, 1234, lt, b) :- codice_corso(3027), 
                                matricola_docente(1234), 
                                laurea(lt), taf(b).

\end{lstlisting}


\subsection{Docenti a contratto}\label{sec:rules-docenti-contratto}

Il file \texttt{docenti\_a\_contratto.lp} contiene informazioni relative ai docenti a 
contratto, una categoria utilizzata come risorsa di emergenza nei casi in cui i vincoli 
ministeriali non possono essere soddisfatti con i docenti a tempo determinato o 
indeterminato. Questi docenti sono identificati dalla fascia \texttt{c} e vengono 
trattati come \textit{jolly}, ovvero risorse flessibili da impiegare per garantire la 
copertura minima dei corsi universitari.

Essendo considerati come \textit{jolly}, i docenti a contratto non sono associati a 
un SSD specifico o a un corso particolare, ma possono essere assegnati liberamente per 
colmare le lacune nei corsi che non soddisfano i requisiti minimi con i docenti delle 
altre fasce. Il file \texttt{docenti\_a\_contratto.lp} è strutturato in modo minimale, 
definendo la fascia \texttt{c} e un unico predicato \texttt{jolly/1}, che indica la 
possibilità di assegnare tali docenti in modo trasversale a qualsiasi corso.

Nel modello ASP proposto, i docenti a contratto vengono inclusi come possibili garanti, 
ma il loro utilizzo è regolato e penalizzato nel processo di ottimizzazione, che vedremo
meglio nella Sezione~\ref{sec:priorita}. 

\begin{lstlisting}[language=prolog, caption={Esempio struttura dati di \texttt{docenti\_a\_contratto.lp}.}]
 %  SEZIONE: FASCIE
 fascia(c).

 jolly(1..5).
\end{lstlisting}


\subsection{Limiti ministeriali}\label{sec:rules-ministeriale}

Il file \texttt{ministeriale.lp} contiene i dati relativi ai requisiti ministeriali per 
ogni corso di laurea, con particolare riferimento al numero minimo di garanti richiesti 
per ciascun corso. Ogni corso ha associato un insieme di valori che determinano i vincoli 
di assegnazione dei docenti, inclusi i docenti a tempo indeterminato e tempo determinato,
e il numero massimo di docenti a contratto.

Le regole \texttt{ministeriale/5} sono calcolate dinamicamente durante la fase di 
preprocessing. In particolare, questi valori sono determinati sulla base del numero di 
studenti iscritti a ciascun corso e grazie all'applicazione della formula della \textit{W}, 
come illustrato nella Tabella~\ref{tab:formula-w}, la quale tiene conto delle specifiche 
necessità di copertura di ciascun corso. Nello specifico, la $W$ viene calcolata nel 
seguente modo:
$$
W = \frac{
        \text{Studenti iscritti al corso}
    }
    {
        \text{Massimo teorico di iscritti al corso}
    }
    - 1
$$

Di seguito un esempio della struttura del file per il corso di Informatica, per 
il quale sono richiesti almeno 9 garanti, di cui 5 a tempo indeterminato, 4 a tempo 
determinato, e un massimo di 2 docenti a contratto. 

\begin{lstlisting}[language=prolog, caption={Esempio struttura dati di \texttt{ministeriale.lp}.}]    
 %  SEZIONE: Garanti minimi per corso (codice_corso, minimo_complessivo, 
 %                                     docenti_ti, docenti_td, 
 %                                     max_docenti_contratto)
 ministeriale(3027, 9, 5, 4, 2).
\end{lstlisting} 

\begin{table}[h]
    \centering
    \renewcommand{\arraystretch}{1.5}
    \begin{tabular}{|l|c|c|}
    \hline
    \textbf{Qualifica} & \textbf{\# base} & \textbf{\# effettivo} \\
    \hline
    Docenza di riferimento & 9 & $\lfloor 9 \times (1+w) \rfloor$ \\
    \hline
    Docenti a tempo indeterminato & 5 & $\lfloor 5 \times (1+w) \rfloor$ \\
    \hline
    Docenti a contratto & 2 & $\lfloor 2 \times (1+w) \rfloor$ \\
    \hline
    \end{tabular}
    \caption{Formule per il calcolo del numero di docenti di riferimento in base alla numerosità degli studenti.}
    \label{tab:formula-w}
\end{table}

\subsection{Generazione dei garanti}\label{sec:garanti}

La generazione dei \textit{garanti} rappresenta un elemento chiave nella modellazione ASP 
sviluppata per questo progetto. In particolare, è stata posta attenzione alla distinzione 
tra i docenti appartenenti a diverse fasce contrattuali, separando i docenti 
\textbf{non a contratto} da quelli \textbf{a contratto}, i quali vengono utilizzati 
esclusivamente come risorsa di emergenza.
Inoltre, ogni docente può essere conteggiato una sola volta oppure, al massimo, essere 
assegnato a due corsi distinti con un peso pari a $0.5$ su ciascun corso. 

Per soddisfare quest'ultimo requisito, il modello ASP proposto utilizza una gestione 
basata sui pesi, i quali rappresentano il livello di impegno del docente come garante:
\begin{itemize}
    \item Un peso di \textbf{10} indica che il docente è assegnato al 100\% su un singolo corso.
    \item Un peso di \textbf{5} rappresenta un docente che divide il proprio impegno al 50\% su due corsi diversi.
\end{itemize}

\begin{lstlisting}[language=prolog, caption=Generazione dei pesi.]
% Predicati di base per definire i pesi possibili.
peso(5).
peso(10).
\end{lstlisting}

\paragraph{Docenti non a contratto.}
I docenti \textit{non a contratto}, appartenenti alle fasce \textit{tempo determinato} 
(\texttt{td}) e \textit{tempo indeterminato} (\texttt{ti}), costituiscono la base principale 
per l'assegnazione dei garanti. La regola ASP proposta genera i possibili garanti per ciascun corso, 
escludendo esplicitamente i docenti a contratto (\texttt{c}).

\begin{lstlisting}[language=prolog, caption=Generazione dei garanti non a contratto.]
Minimo{
      garante(Docente, Corso, Peso, Fascia) :
            peso(Peso),
            fascia(Fascia),
            Fascia != c,

            % Docente 1
            cattedra(Corso, Docente, _, _),
            docente(Docente, Fascia, _),
            
            % Docente 2
            cattedra(Corso, Docente2, _, _),
            docente(Docente2, Fascia, _),

            Docente >= Docente2
}Massimo :-
      ministeriale(Corso, MinimoComplessivo, _, _, MassimoDocentiContratto),
      Minimo = MinimoComplessivo - MassimoDocentiContratto,
      Massimo = MinimoComplessivo.
\end{lstlisting}

In questa regola, il numero minimo di garanti non a contratto per ciascun corso (\texttt{Minimo}) 
viene calcolato sottraendo il numero massimo di docenti a contratto consentiti
dal numero minimo complessivo di garanti richiesto per il corso. 
Questo approccio permette di definire un sottoinsieme minimo di garanti che devono necessariamente 
appartenere alle fasce \texttt{td} e \texttt{ti}. 

Il numero massimo di garanti non a contratto (\texttt{Massimo}), invece, coincide con il valore 
minimo dei garanti richiesti per quel corso, garantendo che la somma totale dei garanti sia 
conforme ai requisiti ministeriali. Inoltre, è stata fissata una relazione d'ordine sul numero di 
matricola (\texttt{Docente >= Docente2}): in questo modo, si vanno ad eliminare le possibili 
permutazioni in fase di generazione.

\paragraph{Docenti a contratto.}
I docenti \textit{a contratto} vengono considerati come una risorsa ausiliaria, identificati tramite 
il predicato \texttt{jolly/1}. A differenza dei docenti non a contratto, i docenti a contratto possono 
essere assegnati a qualsiasi corso, indipendentemente dall'SSD. Tuttavia, il loro utilizzo è regolato 
da un limite massimo per corso, specificato dalla normativa ministeriale, come mostrato nel frammento
di codice seguente.

\begin{lstlisting}[language=prolog, caption=Generazione dei garanti a contratto.]
{     
      garante(Docente, Corso, 10, c) :
            jolly(Docente),
            codice_corso(Corso),
            peso(10),

            jolly(Docente2),
            codice_corso(Corso),
            peso(10),

            Docente > Docente2
}Massimo :- ministeriale(Corso, _, _, _, Massimo).
\end{lstlisting}

Questa regola garantisce che il numero di docenti a contratto utilizzati come garanti non ecceda il 
valore massimo consentito, specificato dal valore \texttt{Massimo} associato al corso. Anche in questo 
caso, viene utilizzata una relazione d'ordine in fase di generazione.


\subsection{Vincoli}\label{sec:constraints}

I vincoli definiti in questo modello costituiscono la struttura portante del sistema, garantendo 
che le soluzioni generate siano coerenti con i requisiti ministeriali e istituzionali. 
Tutti i vincoli introdotti sono \textbf{strong}, il che significa che ogni soluzione che li 
viola viene automaticamente esclusa. 

Per garantire che ogni corso soddisfi i requisiti ministeriali, il numero di docenti assegnati 
come garanti è soggetto a controlli rigorosi. Innanzitutto, i garanti a tempo indeterminato 
(\texttt{ti}) devono essere almeno pari al minimo richiesto per ciascun corso. Allo stesso modo, 
il numero di garanti a tempo determinato (\texttt{td}) non deve superare il massimo consentito. 
Questi vincoli assicurano che il modello rispetti le linee guida istituzionali senza 
sovraccaricare un'unica tipologia di docenti.

\begin{lstlisting}[language=prolog, caption={Vincoli su garanti a tempo indeterminato e determinato.}]
 :- conta_garanti_indeterminato(Corso, Numero),
      ministeriale(Corso, Minimo, Minimo_ind, _, _),
      Numero < Minimo_ind.

 :- conta_garanti_determinato(Corso, Numero),
      ministeriale(Corso, _, _, Massimo_det, _),
      Numero > Massimo_det.
\end{lstlisting}

Il modello include vincoli che controllano i pesi assegnati ai garanti. La somma totale dei 
pesi per ciascun corso deve essere almeno 10 volte il numero minimo di garanti richiesti, 
poiché un docente è considerato \textit{effettivo} solo se il suo peso totale raggiunge il valore 
massimo di 10. Inoltre, tale somma deve essere un multiplo di 10 per garantire la coerenza 
nell'assegnazione del garante stesso.

\begin{lstlisting}[language=prolog, caption={Vincoli sui pesi assegnati ai garanti.}]
 :- somma_pesi_corso(Corso, Somma),
      ministeriale(Corso, Minimo, _, _, _),
      Min = 10 * Minimo,
      Somma < Min.

 :- somma_pesi_corso(Corso, Somma),
      Modulo = Somma \ 10,
      Modulo != 0.
\end{lstlisting}

Per evitare conflitti e garantire una distribuzione equa del carico di lavoro, ogni docente 
può essere associato ad un massimo di due corsi, ma con una somma totale di pesi non superiore 
a 10. Inoltre, un docente non può essere associato allo stesso corso con pesi differenti che, 
sommati, superano 10.

\begin{lstlisting}[language=prolog, caption={Vincoli sulla distribuzione dei pesi per docente.}]
 :- garante(Docente, Corso1, Peso1, Fascia),
      garante(Docente, Corso2, Peso2, Fascia),
      Somma = Peso1 + Peso2,
      Somma > 10,
      Fascia != c,
      Corso1 != Corso2.

 :- garante(Docente, Corso1, Peso1, Fascia),
      garante(Docente, Corso1, Peso2, Fascia),
      Somma = Peso1 + Peso2,
      Somma > 10,
      Fascia != c,
      Peso1 != Peso2.
\end{lstlisting}

Per mantenere la coerenza tra i garanti e gli SSD caratterizzanti di ciascun corso, almeno 
il 50\% dei garanti deve afferire all'SSD del corso. In caso contrario, la soluzione viene esclusa.

\begin{lstlisting}[language=prolog, caption={Vincolo sui garanti afferenti al SSD caratterizzante.}]
 :- numero_garanti_che_afferiscono(Aff, Corso),
      numero_garanti_di_riferimento(Tot, Corso),
      2 * Aff <= Tot.
\end{lstlisting}

Infine, il numero totale di docenti a contratto assegnati a ciascun corso non può superare il 
limite massimo specificato dai requisiti ministeriali.

\begin{lstlisting}[language=prolog, caption={Vincolo sul numero massimo di docenti a contratto.}]
 :- jolly_per_corso(Corso, Numero),
      ministeriale(Corso, _, _, _, Massimo_numero_docenti_a_contratto),
      Numero > Massimo_numero_docenti_a_contratto.
\end{lstlisting}





\subsection{Gestione delle priorità}\label{sec:priorita}
La gestione delle priorità è stata un aspetto fondamentale nel nostro progetto, poiché 
miravamo a rispettare scrupolosamente i vincoli ministeriali, garantendo al contempo 
flessibilità per gestire situazioni particolari in cui tali vincoli non fossero 
pienamente soddisfacibili.

Per organizzare le priorità dei garanti, abbiamo deciso di basarci, in seguito ad un 
colloquio con l'ufficio didattico, principalmente sulla loro tipologia contrattuale. 
Il nostro obiettivo era quello di massimizzare l'impiego di docenti a tempo determinato 
(i.e., ricercatori) e a tempo indeterminato (i.e., professori associati e ordinari), prima 
di ricorrere ai docenti a contratto, che sono utilizzati solo quando strettamente necessario. 
Per ottimizzare la gestione, abbiamo preferito che i docenti assumessero il ruolo di 
garanti per un solo corso, evitando situazioni in cui un docente fosse impegnato come 
garante per più corsi contemporaneamente al 50\%.

\begin{lstlisting}[language=prolog, caption=Gestione delle priorità dei docenti.]
 % Docenti a tempo determinato (ricercatori)
 #maximize { 50 : garante(_, _, _, td) }.

 % Docenti a tempo indeterminato
 #maximize { 40 : garante(_, _, _, ti) }.

 % Docenti a contratto
 #maximize { 32 : garante(_, _, _, c) }.

 % Docenti con peso 10
 #maximize { 25 : garante(_, _, Peso, _), Peso = 10 }.

 % Minimizzare i docenti con peso 5
 #minimize { 100, Docente : garante(Docente, _, Peso, _), Peso = 5 }.
\end{lstlisting}

Inoltre, abbiamo dato priorità ai docenti che insegnano corsi fondamentali per i 
corsi di laurea, distinguendo tra gli insegnamenti di base, identificati dal TAF A, 
e gli insegnamenti caratterizzanti, identificati dal TAF B.

\begin{lstlisting}[language=prolog, caption=Gestione delle priorità del TAF.]
 % Garanti che hanno insegnamenti di base
 #maximize { 25 : garante(Docente, Corso, _, _), 
                  cattedra(Corso, Docente, _, TAF), 
                  TAF = a}.

 % Garanti che hanno insegnamenti caratterizzanti
 #maximize { 18 : garante(Docente, Corso, _, _), 
                  cattedra(Corso, Docente, _, TAF), 
                  TAF = b}.

 % Garanti che hanno insegnamenti affini
 #maximize { 10 : garante(Docente, Corso, _, _), 
                  cattedra(Corso, Docente, _, TAF), 
                  TAF = c}.
\end{lstlisting}

Per ottimizzare l'assegnazione dei garanti in base alla loro competenza, sono stati 
privilegiati i docenti il cui l'SSD corrisponde a quello caratterizzante del corso.
In questo modo, ci siamo assicurati che i garanti siano adeguatamente allineati con 
le competenze richieste dai corsi.

\begin{lstlisting}[language=prolog, caption=Preferenza dei garanti con macrosettore coerente a quello del corso.]
% Ottimizzare i garanti con SSD caratterizzante
#maximize { 20 : garante(Docente, Corso, _, _), 
                 docente(Docente, _, SettoreSSD, _), 
                 corso(Corso, _, SettoreSSD, _) }.
\end{lstlisting}

Per mantenere l'equilibrio tra la qualità e l'efficacia del sistema, abbiamo 
minimizzato il numero di garanti assegnati a ciascun corso, penalizzando le situazioni 
in cui il numero di garanti supera il minimo richiesto. Questo vincolo ci ha permesso 
di ottimizzare le risorse, evitando l'assegnazione di più docenti di quanto fosse necessario.

\begin{lstlisting}[language=prolog, caption=Minimizzazione dei garanti per corso di laurea.]
 % Minimizzo il numero di garanti per ogni corso
 #minimize { Penalita : garanti_per_corso(Corso, Numero),
                        Penalita = Numero * 10 }.
\end{lstlisting}

Infine, abbiamo introdotto un ulteriore criterio di ottimizzazione per privilegiare i 
garanti che ricoprono il ruolo di presidente del loro corso di laurea. 

\begin{lstlisting}[language=prolog, caption=Massimizzazione dei presidenti come garanti.]
 % Massimizza i presidenti che sono garanti nel loro corso di laurea
 #maximize { 50, Docente, Corso : garante(Docente, Corso, _, _), 
                                  presidente(Docente, Corso) }.
\end{lstlisting}