\section{ASP}\label{sec:asp}

In questa sezione approfondiamo l'organizzazione del progetto, evidenziando 
come sia stato strutturato per garantire chiarezza e modularità. 
Per facilitare l'ordine visivo e semplificare eventuali ispezioni manuali, 
il progetto è stato suddiviso in più file sorgenti, ciascuno dedicato a un 
aspetto specifico del dataset e delle regole.

Abbiamo separato i fatti e le regole in due file principali: uno dedicato 
ai docenti (Sezione~\ref*{sec:rules-docenti}) e un altro focalizzato sulle 
coperture (Sezione~\ref*{sec:rules-coperture}). Per maggiore praticità, è 
stato creato un file specifico per i docenti a contratto 
(Sezione~\ref*{sec:rules-docenti-contratto}) e un altro per i limiti 
ministeriali (Sezione~\ref*{sec:rules-ministeriale}).

Il file principale, il main, integra tutte le regole relative ai garanti, 
inclusi i vincoli e le priorità definite per l'ottimizzazione del sistema 
(Sezione~\ref*{sec:constraints}).

È importante sottolineare che tutti questi file vengono generati dinamicamente 
dal programma di preprocessing scritto in Python. Questa fase, gestita dal 
file main del preprocessing, automatizza la creazione delle regole e dei 
fatti ASP in base ai dati forniti, garantendo così una configurazione 
personalizzata e facilmente aggiornabile. 


\subsection{Docenti}\label{sec:rules-docenti}



\begin{lstlisting}[language=Python, caption={Esempio di codice Python}]
def saluta(nome):
    print(f"Ciao, {nome}!")

saluta("Mondo")
\end{lstlisting}


\subsection{Coperture}\label{sec:rules-coperture}
\subsection{Docenti a contratto}\label{sec:rules-docenti-contratto}
\mmerenda{spiegare qui che li usiamo come jolly}
\subsection{Ministeriale}\label{sec:rules-ministeriale}


\subsection{Vincoli}\label{sec:constraints}
\subsection{Gestione delle priorità}\label{sec:priorita}

La gestione delle priorità è stata un elemento cruciale di questo progetto.
Abbiamo voluto fornire un sistema che rispettasse i requisiti ministeriali e
che fosse al contempo flessibile, per affrontare situazioni particolari in cui
tali requisiti non potessero essere pienamente soddisfatti.

Per prima cosa abbiamo deciso di gestire le priorità dei garanti in base alla loro tipologia contrattuale.
Per farlo abbiamo di massimizzato, in ordine, l'utilizzo dei ricercatori, dei professori associati/ordinari e dei docenti a contratto;
preferendo che i docenti facciano da garanti al 100\% su un singolo corso, penalizzando la casistica
del 50\% su due corsi differenti.

\begin{lstlisting}[language=prolog, caption=Gestione delle priorità dei docenti]
% Massimizzare i docenti a tempo determinato (ricercatori)
#maximize { 50 : garante(_, _, _, td) }.

% Massimizzare i docenti a tempo indeterminato
#maximize { 40 : garante(_, _, _, ti) }.

% Minimizzare i docenti a contratto
#maximize { 32 : garante(_, _, _, c) }.

% Massimizzare i docenti con peso 10
#maximize { 25 : garante(_, _, Peso, _), Peso = 10 }.

% Minimizzare i docenti con peso 5
#minimize { 100, Docente : garante(Docente, _, Peso, _), Peso = 5 }.
\end{lstlisting}

Abbiamo deciso di favorire la scelta di garanti che insegnano corsi fondamentali per
i corsi di laurea, classificati come TAF A (insegnamenti di base) e TAF B
(insegnamenti caratterizzanti), rispetto agli altri insegnamenti.

\begin{lstlisting}[language=prolog, caption=TAF]
#maximize { 25 : garante(Docente, Corso, _, _), 
                cattedra(Corso, Docente, _, TAF), 
                TAF = a}.
#maximize { 18 : garante(Docente, Corso, _, _), 
                cattedra(Corso, Docente, _, TAF), 
                TAF = b}.
#maximize { 10 : garante(Docente, Corso, _, _), 
                cattedra(Corso, Docente, _, TAF), 
                TAF = c}.
\end{lstlisting}

Per garantire l’aderenza tra le competenze dei garanti e i corsi assegnati,
abbiamo dato la priorità ai docenti il cui macrosettore scientifico-disciplinare (SSD)
è coerente con quello caratterizzante del corso.

\begin{lstlisting}[language=prolog, caption=Garanti con macrosettore coerente a quello del corso.]
% Ottimizzare i garanti con SSD caratterizzante
#maximize { 20 : garante(Docente, Corso, _, _), 
                docente(Docente, _, SettoreSSD, _), 
                corso(Corso, _, SettoreSSD, _) }.
\end{lstlisting}

Abbiamo minimizzato il numero di garanti assegnati a ciascun corso, penalizzando
situazioni in cui il numero di garanti supera il minimo richiesto.

\begin{lstlisting}[language=prolog, caption=Minimizzazione dei garanti per corso di laurea.]
% Minimizzo il numero di garanti per ogni corso
#minimize { Penalita : 
            garanti_per_corso(Corso, Numero),
            Penalita = Numero * 10 }.
\end{lstlisting}

Nel modello proposto abbiamo introdotto una serie di vincoli deboli per avere
maggiore flessibilità nella gestione di situazioni in cui non è possibile rispettare
rigorosamente i requisiti ministeriali.

\begin{lstlisting}[language=prolog, caption=Vincoli di flessibilità]
% Vincolo: il numero di docenti a tempo indeterminato per ogni corso deve
% essere almeno pari al minimo ministeriale richiesto.
% Se il numero e' inferiore al minimo, la soluzione viene scartata.
% ! weak constraint
:~ conta_docenti_indeterminato(Corso, Numero),
        ministeriale(Corso, Minimo, Minimo_ind, _, _),
        codice_corso(Corso),
        Numero < Minimo_ind.
        [Numero * 10@1]

% Vincolo: il numero di docenti a tempo determinato per ogni corso non deve
% essere superiore al massimo ministeriale richiesto.
% Se il numero e' superiore al massismo, la soluzione viene scartata.
% Penalizzazione per ricercatori che superano il limite consentito
% ! weak constraint
:~ conta_docenti_determinato(Corso, Numero),
        ministeriale(Corso, _, _, Massimo_det, _),
        Numero > Massimo_det,
        Penalita = Numero - Massimo_det.
        [Penalita * 50@1]
% Vincolo: il numero di docenti che afferiscono al SSD del corso deve rappresentare almeno il 50%
% del numero totale di docenti di riferimento.
% Se la condizione non e' soddisfatta, la soluzione viene scartata.
% ! weak constraint
:~ numero_docenti_che_afferiscono(Aff), 
        numero_docenti_di_riferimento(Tot),
        2 * Aff <= Tot. 
        [Tot - 4 * Aff@1]
\end{lstlisting}

\begin{lstlisting}[language=prolog, caption=Minimizzazione dei garanti per corso di laurea.]
% Vincolo debole per minimizzare il numero di garanti effettivi rispetto al minimo richiesto.
% Penalizza le soluzioni in cui il numero di garanti effettivi
% (calcolato da `garanti_per_corso`) non coincide con il numero minimo
% stabilito dai requisiti ministeriali per il corso.
:~ garanti_per_corso(Corso, Numero),
        ministeriale(Corso, Minimo, _, _, _),
        Numero != Minimo.
        [Numero * Minimo@1]
\end{lstlisting}

\subsection{Ottimizzzazioni}\label{sec:optimizations}